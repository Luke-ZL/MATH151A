\documentclass{article}
\author{Zixuan Lu \\ 304990072}
\title{\textbf{Homework 1}}
\usepackage{amsmath}
\usepackage{amssymb}
\usepackage[margin=1 in]{geometry}
\begin{document}
\maketitle
\paragraph{Problem 1}
\subparagraph*{(a)} Since $f(0.5)=0.25-0.35=-0.1 < 0$, $f(1)=1-0.7=0.3>0$,by \textit{Intermediate Value Theorem}, $\exists d\in[0.5,1]$ s.t. $f(d)=0$. \\
$\Rightarrow f$ has at least 1 root. - - - - - \textcircled{1}\\


Assume there are two roots  $a,b\in[0.5,1]$ s.t. $f(a)=f(b)=0$.Hence, by \textit{Rolle's Theorem}, $\exists c\in[a,b]$ s.t. ${f}'(c) = 0$ \\
However, ${f}'(x)=2x-0.7>0$ for $x\in[0.5,1] \Rightarrow contradiction$ \\
$\Rightarrow f$ cannot have two or more roots for $x\in[0.5,1]$. - - - - - \textcircled{2}\\ \\
By \textcircled{1} and \textcircled{2}, $f(x)=x^2-0.7x, x\in[0.5,1]$ has exactly one root.
\\
\subparagraph*{(b)} Since $|P_{n}-P|\leq\dfrac{b-a}{2^n},n\geq1$. Let b=1, a=0.5\\
\begin{equation}\nonumber
\begin{split}
\dfrac{1-0.5}{2^n} &< 10^-5 \\
2^n & >5\times10^4 \\
n &> 15,29 \\
n &= 16 \\
\end{split}
\end{equation} 
Hence, 16 iterations are required.
\paragraph{Problem 2}
\subparagraph*{\textcircled{1}} If $f(a)=a$ or $f(b)=b$, then there exists at least one fixed point at the endpoint of the interval.
\subparagraph*{\textcircled{2}} If not, since $f(x)\in[a, b], f(a)>a, f(b)<b$. Let $h(x)=x-f(x) \Rightarrow h(a)<0, h(b)>0$. The \textit{Intermediate Value Theorem} implies $\exists p\in(a,b)$ s.t. $h(p)=0 \Rightarrow g(p)=p$, there exists at least one fixed point. \\
\\
$\Rightarrow f$ has at least a fixed point on $[a,b]$
\newpage
\paragraph{Problem 3}
\subparagraph*{(a)}
\begin{equation}\nonumber
\begin{split}
p_{1} &= \dfrac{p_{0}^2+3}{2p_{0}}=\dfrac{9+3}{2\times3}=2 \\
p_{2} &= \dfrac{p_{1}^2+3}{2p_{1}}=\dfrac{4+3}{2\times2}=\dfrac{7}{4}
\end{split}
\end{equation} 
\subparagraph*{(b)}
\begin{equation}\nonumber
\begin{split}
p_{n}&=\dfrac{p_{n}^2+3}{2p_{n}} \\
2p_{n}^2 &= p_{n}^2+3\\
p_{n}^2 &= 3 \\
p_{n}&=\pm\sqrt{3}
\end{split}
\end{equation} 
$\Rightarrow$ all possible limits are $\pm\sqrt{3}$.
\subparagraph*{(c)} \textit{Newton's Method} is the functional iteration technique with \\
\begin{equation}\nonumber
\begin{split}
p_{n} = g(p_{n}-1)&=p_{n-1}-\dfrac{f(p_{n-1})}{{f}'(p_{n-1})} \\
&=p_{n-1} - \dfrac{p_{n-1}^2-3}{2p_{n-1}} \text{ for } f(x) = x^2-3 \\
&=\dfrac{2p_{n-1}^2-p_{n-1}^2+3}{p_{n-1}} \\
&=\dfrac{p_{n-1}^2+3}{2p_{n-1}} \text{ (shown)}
\end{split}
\end{equation} 
\paragraph{Problem 4}
\subparagraph*{(a)}
\begin{equation}\nonumber
\begin{split}
p_{n} &= p_{n-1} - \dfrac{f(p_{n-1})(p_{n-1}-p_{n-2}}{f(p_{n-1})-f(p_{n-2})} \\
p_{2} &= 3-\dfrac{(3^2-3)\cdot(3-\dfrac{1}{2})}{3^2-3-(\dfrac{1}{2})^2+3} \\
&=3-\dfrac{15}{8.75} \\
&=\dfrac{9}{7} \\
p_{3} &= \dfrac{9}{7} - \dfrac{[(\dfrac{9}{7})^2-3]\cdot(\dfrac{9}{7}-3)}{(\dfrac{9}{7})^2-3-3^2+3} \\
&= \dfrac{8}{5}
\end{split}
\end{equation} 
\newpage
\subparagraph*{(b)}
\begin{equation}\nonumber
\begin{split}
q_{0} &= (\dfrac{1}{2})^2 - 3 = -2.75 \\
q_{1} &= 3^2 - 3 = 6 \\
p_{2} &= p_{1}-q_{1}(p_{1}-p_{0})/(q_{1}-q_{0}) \\
&= 3 - 6(3-\dfrac{1}{2})/(6-(-2.75)) = \dfrac{9}{7} \\
q_{2} &= (\dfrac{9}{7})^2 - 3 =-\dfrac{66}{49}\\
p_{3} &= \dfrac{9}{7} - (-\dfrac{66}{49})\dfrac{\dfrac{9}{7}-3}{-\dfrac{66}{49}-6} = \dfrac{8}{5}
\end{split}
\end{equation}
\end{document}
