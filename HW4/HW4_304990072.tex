\documentclass{article}
\author{Zixuan Lu \\ 304990072}
\title{\textbf{Homework 4}}
\usepackage{amsmath}
\usepackage{amssymb}
\usepackage{physics}
\usepackage[margin=1 in]{geometry}
\begin{document}
\maketitle
\paragraph{Problem 1}
\subparagraph*{(a)} 
\begin{equation}\nonumber
\begin{split}
L_1(x)&=\frac{(x-0)(x-1)}{(-1-0)(-1-1)} = \frac{x^2-x}{2} \\
L_2(x)&=\frac{[x-(-1)](x-1)}{[0-(-1)](0-1)} = 1-x^2 \\
L_3(x)&=\frac{[x-(-1)](x-0)}{[1-(-1)](1-0)} = \frac{x^2+x}{2} \\
h(x) &= P(x) = \sum_{k=0}^{3} f(x_k)L_k(x) \\
&= \frac{x^2-x}{2}f(-1) + (1-x^2)f(0) + \frac{x^2+x}{2}f(1)
\end{split}
\end{equation}

\subparagraph*{(b)}
$$E(x)=\frac{f^{(3)}(\xi(x))}{3!}\prod_{i=0}^{2} (x-x_i)$$ 

\subparagraph*{(c)}
\begin{equation}\nonumber
\begin{split}
\int_{-1}^{1} h(x)\ dx &= \int_{-1}^{1} \frac{x^2-x}{2}f(-1) + (1-x^2)f(0) + \frac{x^2+x}{2}f(1)\ dx \\
&= \frac{1}{2}\int_{-1}^{1} (x^2-x)f(-1) + (2 - 2x^2)f(0) + (x^2 + x)f(1)\ dx \\
&= \frac{1}{2}f(-1)\int_{-1}^{1} x^2-x\ dx + f(0)\int_{-1}^{1}1-x^2\ dx + \frac{1}{2}f(1)\int_{-1}^{1} x^2+x\ dx \\
&= \frac{1}{2}f(-1)[\frac{x^3}{3}-\frac{x^2}{2}]_{-1}^1 + f(0)[x-\frac{x^3}{3}]_{-1}^1 + \frac{1}{2}f(1)[\frac{x^3}{3}+\frac{x^2}{2}]_{-1}^1\\
&= \frac{1}{3}f(-1) + \frac{4}{3}f(0) + \frac{1}{3}f(1)
\end{split}
\end{equation}

\subparagraph*{(d)}
Yes. This is because when $f$ is a polynomial of degree less than or equal to 2, the error term stated in part (b) is $0$ as $f^{(3)}(\xi(x))$ is 0. This makes $\int_{-1}^{1} h(x)\ dx = \int_{-1}^{1} f(x)\ dx$.

\subparagraph*{(e)}
\begin{equation}\nonumber
\begin{split}
|\int_{-1}^{1} f(x)\ dx - \int_{-1}^{1} h(x)\ dx| &\leq \frac{1}{6} \int_{-1}^{1} (x-x_0)(x-x_1)(x-x_2)f^{(3)}(\xi(x))\ dx 
\end{split}
\end{equation} 
\newpage
\paragraph{Problem 2}
\subparagraph*{(a)} 
\begin{equation}\nonumber
\begin{split}
\int_{0}^{4} f(x)\ dx &= \frac{h}{3}[f(x_0)+4f(x_1)+f(x_2)]-\frac{h^5}{90}f^{(4)}(\xi)  \\
&\approx \frac{2}{3}[f(0)+4f(2)+f(4)] \\
&= \frac{2}{3}[1+4+1] \\
&= 4
\end{split}
\end{equation}

\subparagraph*{(b)}
\begin{equation}\nonumber
\begin{split}
\int_{1}^{4} f(x)\ dx &= \frac{h}{3}[f(a)+2\sum_{j=1}^{(n/2)-1}f(x_{2j}) +4\sum_{j=1}^{(n/2)}f(x_{2j-1})+f(b)]-\frac{b-a}{180}h^4f^{(4)}(\mu) \\
&\approx \frac{1}{3} [f(0)+2f(2)+4(f(1)+f(3))+f(4)] \\
&=\frac{1}{3}[1+2*1+4*(2+2)+1] \\
&= \frac{20}{3}
\end{split}
\end{equation}
















\end{document} 
